\documentclass{scrartcl}

\usepackage[hidelinks]{hyperref}
\usepackage[none]{hyphenat}
\usepackage{setspace}
\doublespace

%Please include a clear, concise, and descriptive title
\title{Fourth Generation Games Consoles And How The SNES Affecting History}

%Please do not change the subtitle
\subtitle{COMP130 - Game Platform History Essay}

%Please put your student ID in the author field
\author{DH185421}

\begin{document}

\maketitle


\section{Introduction}

This report examines the Super Nintendo Entertainment System (SNES) through the lens of its media reception. Specifically, it will compare how concerns about violence and desensitization have persisted in modern media, despite the considerable technical limitations of the 4th generation console technology. The following sections will firstly be about what the technical aspects of the Super Nintendo Entertainment System is, it will then move on to talk about how the SNES's games have influenced desensitization such as Mortal Kombats progression from censored effects to no censoring at all. Thirdly it will talk about the realism of games and how such things as crashing in a game has no effect on the player.
\section{What is the SNES?}
Released in Japan in 1990 the SNES is a 16 bit games console developed by Nintendo, being a global success the SNES was the best selling 16 bit games console and was still hugely popular through the 32 bit era.~\cite{2}. The SNES had the ability to allow tilling and simulated 3D effects with a pallet of 32, 768 colors as well as a high quality 8 channel audio because of its state of the art graphics and sound co processors.~\cite{1}. The SNES had 128kb of RAM which is mapped to various segments of bus A, and can also be accessed in serial fashion via registers on bus B. The video and audio systems had additional 64kb of RAM for those processors. The central processing unit for the SNES was the Ricoh 5A22; based on the 16 bit CMD/GTE 65c816 it runs at 3.58MHz. The main features of the CPU were circuitry for generating non maskable interrupts on v blank and IRQ interrupts on calculated screen positions.~\cite{5}
\section{Games and desensitization}
Next the SNES sold 49 million units at approximately £150 a unit~\cite{2}. Although it's predecessor sold considerably more units the SNES had the top sales compared to other fourth generation games consoles. Leading to the point addressed by the report, just over 780 games were released for the SNES, the impact these games were having on people during the time were being seen. Studies believe that computer games were indicating “significant differences in aggressive behaviour”~\cite{3}. Although everybody is prone to influence, children are more vulnerable to subliminal messages~\cite{6}. Another point following what James Porter was writing about is desensitization. Desensitization is as the diminished emotional responses to a negative or aversive stimulus after repeated exposure, in this report's case it is referring to repeated exposure through playing video games. Granting that the time it takes for someone to become desensitized can vary depending on the stimulus, you are able to recognise that the list of games from the SNES are less violent compared to the more recent game titles from the Xbox. A prime example is Mortal Kombat, SNES released Mortal Kombat in 1992 but as a family friendly version. In this version the blood was replaced with sweat and the fatalities were less brutal~\cite{8}, on the other hand the Xbox version of Mortal Kombat released in 2015 uses no caution towards violence and gore.

Although there has been many studies on the matter, there has also been many studies that has evidence against video games causing aggression and desensitization such as Gibb, Bailey, Lambirth, and Wilson (1983) which found no relation in a larger study of 12-34 year olds. When Super Mario first came out in 1980s it was seen as a violent game although it had cartoon like characters which would destroy blocks and jump and kill NPCs, today's standards show the Mario franchises as a child's game and is not seen as violent.~\cite{9}
\section{Realism}
 Perhaps because of the advancement in technology it has pressed video games to become more verisimilitude therefore desensitising the players through the realism of the violence. On this point because of the SNESs lack of graphical power, games such as driving sims lacked the verisimilitude needed to simulate a real crash. Players who crash in a game do not get the impact as if they were in an actual car, they'd simply restart the race where as if it were real they'd face much more serious consequences. ~\cite{9} This comes to the point of the advancement in technology, where the SNESs graphics does not give the player the realism needed for them to feel anything from crashing it has in turn pushed future consoles to make games as realistic as possible.The Xboxs graphics are greatly improved than that of the SNES but players are still not affected by the fact they crash even though visually the verisimilitude is absolute.~\cite{9}

\section{Conclusion}

In conclusion on the point set at the start of this report, not only has the SNES pushed advances in technology to become more verisimilitude but it is one of many that has desensitized society. Although not visually as upsetting as the Xbox, for its time many of it's games were seen as violent games and over the years of video games the violence in these games were increasing and developers were pushed to make these scenes as realistic as the hardware allowed it. 


\begin{thebibliography}{}
\bibitem{1}  Bibliography:Technical Details (2011)
http://www.nintendo.co.uk/Corporate/Ni HYPERLINK "http://www.nintendo.co.uk/Corporate/Nintendo-History/Super-Nintendo/Technical-Details/Technical-Details-627042.html" HYPERLINK "http://www.nintendo.co.uk/Corporate/Nintendo-History/Super-Nintendo/Technical-Details/Technical-Details-627042.html" HYPERLINK "http://www.nintendo.co.uk/Corporate/Nintendo-History/Super-Nintendo/Technical-Details/Technical-Details-627042.html"ntendo-History/Super-Nintendo/Technical-Details/Technical-Details-627042.html

\bibitem{2}Bibliography: Nintendo History (2013) Available at: http://www.nintendo.co.uk/Corporate/Nintendo-History/Nintendo-History-625945.html (Accessed: 14 October 2015) In-line Citation: (Nintendo History, 2013)

\bibitem{3}Bibliography: Porter, J. P. and Book, W. F. (eds.) (2007) Journal Of Applied Psychology. United States: Kessinger Publishing. In-line Citation: (Porter and Book, 2007)

\bibitem{4}Bibliography: B Funk, J. (2004) Journal of Adolescence. Jeanne B Funk. In-line Citation: (B Funk, 2004)

\bibitem{5}Bibliography: Documents - Anomie’s SNES Memory Mapping Doc (2008) Available at: http://www.romhacking.net/documents/193/ (Accessed: 25 October 2015). In-line Citation: (Documents - Anomie’s SNES Memory Mapping Doc, 2008)

\bibitem{6}Bibliography: Steinberg, L. and Monahan, K. C. (2009) ‘Age Differences in Resistance to Peer Influence’, 43(6). In-line Citation: (Steinberg and Monahan, 2009)

\bibitem{7} http://www.nintendo.co.jp/ir/library/historical_data/pdf/consolidated_sales_e0912.pdf

\bibitem{8}Bibliography: Kent, S. L. (2001) The Ultimate History of Video Games: From Pong to Pokemon and Beyond.he Story Behind the Craze that Touched Our Lives and Changed the World. 1st edition. Roseville, CA: Crown Publishing Group. In-line Citation: (Kent, 2001)

\bibitem{9}http://www-cs-students.stanford.edu/~geksiong/papers/cs378/cs378paper.pdf
\end{thebibliography}
\end{document}
